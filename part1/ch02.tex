\chapter{A Gentle Start}

\begin{ex}
  Our goal is to show that given a training set
  $S=\{(\x_i, f(\x_i))\}_{i=1}^m\subseteq (\R^d\times \{0, 1\})^m$,
  there exists a polynomial $p_S$ such that $p_S(\x)\geq 0$ if and only if
  $\x\in S$ such that $f(\x)=1$.

  For every $\x_i\in S$, we construct a polynomial
  $Q_i(\mb{t})=(t_1-x_{i,1})^2+\cdots+(t_d-x_{i,d})^2$,
  and note that $Q_i(\mb{t})\geq 0$ with equality if and only if
  $\mb{t}=\x_i$. Therefore,
  \begin{align*}
    p_S(\mb{t})=-\prod_{\substack{i \in [d] \\ f(\x_i) = 1}}Q_i(\mb{t})
  \end{align*}
  has a zero whenever $\mb{t}=\x_i$ for an $i$ such that $f(\x_i)=1$, and is
  negative otherwise.
\end{ex}

\begin{ex}
  We have
  \begin{align*}
    \underset{S|_x \sim \D^m}{\E}\left[L_S(h)\right]
     & = \underset{S|_x \sim \D^m}{\E}\left(\frac{|\{i\in[m] \mid h(x_i)\neq y_i\}|}{m}\right)     \\
     & = \frac{1}{m}\left(\underset{S|_x \sim \D^m}{\E}|\{i\in[m] \mid h(x_i)\neq f(x_i)\}|\right) \\
     & = \frac{1}{m}\sum_{i=1}^m\underset{x\in \D}{\E}|\{i \mid h(x)\neq f(x)\}|                   \\
     & = \D(\{x \mid h(x)\neq f(x) \})                                                             \\
     & = L_{(\D, f)}(h).
  \end{align*}
\end{ex}

\begin{ex}
  \begin{enumerate}
    \item[]
    \item Suppose not, i.e. suppose that $L_S(A(S))>0$ for some training set
          $S$. Since $A(S)$ encloses
          all positive examples in the training set, it follows that there must
          be an erroneously enclosed negative example, $s^-$. Under the
          realizability assumption, there exists a hypothesis $h^*$ such that
          $h^*(s^-)=0$, but $h^*$ of all the positive examples in the training
          set is $1$. Hence, the intersection of the rectangle associated
          with $h^*$ with the rectangle associated with $A(S)$ is strictly
          smaller than $A(S)$, but encloses all positive examples, a
          contradiction to the definition of $A$.
    \item We proceed identically to the hint: we fix a distribution $\D$, we let
          $R^*=R(a_1^*, b_1^*, a_2^*, b_2^*)$ be the rectangle corresponding to the
          true labelling and let $f$ be its corresponding hypothesis. We select an
          $a_1\geq a_1^*$ such that $R_1=R(a_1^*, a_1, a_2^*, b_2^*)$ has mass
          (under $\D$) of $\epsilon/4$, and similarly choose $b_1, a_2, b_2$ so
          that
          $R_2=R(b_1, b_1^*, a_2^*, b_2^*)$,
          $R_3=R(a_1^*, b_1^*, a_2^*, a_2*)$,
          $R_4=R(a_1^*, b_1^*, b_2, b_2^*)$ each have probability $\epsilon/4$.
          \begin{itemize}
            \item []
            \item Note that under the realizability assumption,
                  $R^*$ is a rectangle that contains all positive
                  examples in the training set $S$, and that therefore must
                  contain $R(S)$, the minimal rectangle containing all positive
                  examples. Hence, $R^*\subseteq R(S)$.
            \item Note that $R^*=R(S)\cup R_1\cup R_2\cup R_3\cup R_4$, and that
                  therefore
                  \begin{align*}
                    L_{(\D,f)}(A(S))
                     & =\D(R^*)\setminus \D(R(S))                 \\
                     & \leq \D(R_1) + \D(R_2) + \D(R_3) + \D(R_4) \\
                     & =\epsilon.
                  \end{align*}
            \item Note that the probability that an example drawn from $\D$ is
                  not in $R_i$ is $1-\epsilon/4$, and that therefore the probability
                  that $S$ does not contain any examples from $R_i$ is
                  $(1-\epsilon/4)^m$, which we can be bounded from above by
                  $e^{-m\epsilon/4}$.
            \item Hence, by the union bound, the probability that $S$ does not
                  contain examples from all of $R_1$, $R_2$, $R_3$ and $R_4$ is
                  bounded from above by $4e^{-m\epsilon/4}$,
                  or, solving $4e^{-m\epsilon/4}\leq \delta$ for $m$,
                  \[
                    m\geq \frac{4\log(4/\delta)}{\epsilon},
                  \]
                  as expected.
          \end{itemize}
    \item Note that analogously to the previous section, we can define $2d$
          rectangles in $\R^d$, $R_1,\ldots,R_{2d}$, such that
          $\D(R_i)=\frac{\epsilon}{2d}$ and
          \begin{align*}
            L_{(\D,f)}(A(S))
            =\D(R^*)\setminus \D(R(S))
            \leq \sum_{i=1}^{2d}\D(R_i)
            =\epsilon.
          \end{align*}

          The probability that an example drawn from $\D$ is not in $R_i$ will
          then be $1-\frac{\epsilon}{2d}$, and hence the probability that $S$
          does not contain any examples from $R_i$ will be bounded from above by
          $e^{-m\epsilon/(2d)}$.

          Thus, by the union bound, the probability that $S$ does not contain
          examples from all the $R_i$'s will be bounded from above by
          $2de^{-m\epsilon/(2d)}$, or solving
          $2de^{-m\epsilon/(2d)}\leq \delta$ for $m$,
          \[
            m\geq \frac{2d\log(2d/\delta)}{\epsilon}
            =\frac{2d\log(2d)+2d\log(1/\delta)}{\epsilon}.
          \]
    \item Note that running the algorithm $A$ requires us to select the minimum
          and maximum of each of the $d$ coordinates for all the $m$ examples.
          Hence, the running time is $O(dm)$, but we showed earlier that the
          required number of examples is polynomial in $1/\epsilon$ and
          $\log(1/\delta)$. Thus, the result follows.
  \end{enumerate}
\end{ex}