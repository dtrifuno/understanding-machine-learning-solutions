\chapter{The VC-Dimension}

\begin{ex}
  Suppose not, i.e.\ suppose that $\H'\subseteq \H$, but
  $\VCdim(\H')>\VCdim(H)=n$. Then there exists some set $C\subseteq \X$, with
  $|C|=n+1$ that $\H'$ shatters. However, since $\H'\subseteq \H$, it follows
  that $\H'_C\subseteq \H_C$ and therefore $\H$ also shatters $C$,
  contradicting the assumption that $\VCdim(\H)=n$.
\end{ex}

\begin{ex}
  \begin{enumerate}
    \item[]
    \item Let $n$ be the cardinality of the domain set $\X$. Then, the
          VC-dimension of $\H^\X_{=k}$ is $\min(k, n-k)$.

          First, note that it is never possible to shatter a set of $k+1$
          points, since there is no $h\in\H^\X_{=k}$ such that $h(x)=1$ for
          $k+1$ different points. Likewise, it is never possible to shatter a
          set of $n-k+1$ points, since if a hypothesis assigns the value $1$ to
          exactly $k$ elements, it must assign the value $0$ to exactly $n-k$
          elements, and hence there is no hypothesis corresponding to the
          constant $0$ function.

          Next, suppose that $k\leq n-k$. Let $C=\{p_i\}_{i=1}^k$ be a subset of
          $\X$, and note that for any $v\in\{0,1\}^k$, we can construct a
          hypothesis $h(p_i)=v_i$, since we know that at most $k$ of the $v_i$'s
          are going to be $1$, and that there are at least $k$ more elements in
          $X\setminus C$ that we can map to $1$ in order to satisfy the
          condition on the hypothesis space.

          Finally, suppose that $n-k\leq k$. Let $D=\{p_i\}_{i=1}^{n-k}$ be a
          subset of $\X$, and note that any function $h:D\to \{0,1\}^{n-k}$ is
          in $H^\X_{=k}$, since $|D|=n-k\leq k$, and hence $h$ can assign the
          value $1$ enough times and $|\X\setminus D|=n-(n-k)=k\geq n-k$, and
          hence $h$ can assign the value $0$ enough times.
    \item Let $n$ be the cardinality of the domain set $\X$. Then
          \begin{align*}
            \VCdim(\H^\X_{\text{at-most-$k$}})=\begin{cases}
              n & \text{if $n\leq 2k+1$}, \\
              k & \text{otherwise}.
            \end{cases}
          \end{align*}

          First, note that if $n\leq 2k+1$, then for every $f\in \X^{\{0, 1\}}$,
          it is either going to take on the value $1$ at most $k$ times, or the
          value $0$ at most $k$ times, and hence is in
          $\H^\X_{\text{at-most-$k$}}$.

          Next, suppose that $n>2k+1$. Note that it is never possible to shatter
          a set $C$ of $k+1$ points, since a function that returns $1$ for the
          $k+1$ elements of $C$ will have to return $0$ for the remaining
          $|\X\setminus C|>k$ elements, and thus is not of the hypothesis class.

          Finally, note that if $C=\{p_i\}_{i=1}^k$, we can always construct a
          hypothesis $h(p_i)=v_i$ for any $v\in \{0,1\}^k$, by specifying that
          $h(x)=0$ if $x\in \X\setminus C$ and observing that $h$ then assigns
          the value $1$ at most $k$ times.
  \end{enumerate}
\end{ex}

\begin{ex}
  Note that there is a one-to-one correspondence between hypotheses in
  $\H_{n-\text{parity}}$ and subsets of $[n]$, by mapping each hypothesis $h_I$
  to $I\subseteq [n]$. Hence, $|\H_{n-\text{parity}}|=2^n$, and it follows by
  the VC-dimension bound on finite classes that
  $\VCdim(\H_{n-\text{parity}})\leq n$.

  To show that the VC-dimension is $n$, it remains to show that the hypothesis
  class can shatter a set of cardinality $n$. Let $C=\{e_i\}_{i=1}^n$, the set
  of unit vectors, and note that for any $v\in \{0, 1\}^n$, if we define
  $I=\{i\in [n]\mid v_i = 1 \}$, the hypothesis $h_I$ maps $e_i$ to $v_i$.
  Hence, $\H_{n-\text{parity}}$ shatters $C$.
\end{ex}

\begin{ex}
\end{ex}

\begin{ex}
  First, we produce a set $C=\{p_m\}_{m=1}^{2d}$ that $\H_{\text{rec}}^d$
  shatters, by defining every coordinate of $p_{2i}$ to be zero, except the
  $i$th coordinate which we take to be $1$, and letting every coordinate of
  ${p_{2i+1}}$ be zero, except the $i$th coordinate which we take to be $-1$.
  Note that hypotheses in $\H_{\text{rec}}^d$ are parametrized by vectors
  $(t_{1,\text{low}}, t_{1,\text{high}},\ldots,t_{d,\text{low}},t_{d,\text{high}})$
  with $t_{i,\text{low}}<t_{i,\text{high}}$ for all $i$, where
  $t_{i,\text{low}}$ and $t_{i,\text{high}}$ denote, respectively, the lowest
  and highest value of the $i$th coordinate of the rectangle. Then, given any
  $v\in\{0, 1\}^{2d}$, we can construct the axis-aligned rectangle corresponding
  to the vector given by $t_{i,\text{low}}=-v_{2i+1}-\frac{1}{2}$ and
  $t_{i,\text{high}}=v_{2i}+\frac{1}{2}$ and note that it produces the desired
  labelling.

  Next, consider a set $D$ of $2d+1$ points and take the minimal axis-aligned
  rectangle that contains it. Note that we can parametrize this rectangle by
  taking the minimum and maximum of coordinates of points in our set along each
  axis, but that since there are $2d$ such parameters, and $2d+1$ points, by the
  pigeonhole principle there must exist a point $p$ that we can exclude from the
  set, but still have the same minimum rectangle over the remaining points.
  Hence, it is not possible for a hypothesis in $\H_{\text{rec}}^d$ to label all
  of $D\setminus\{p\}$ with $1$, but $p$ with $0$.
\end{ex}

\begin{ex}
\end{ex}

\begin{ex}
  \begin{enumerate}
    \item[]
    \item For any real number $r$, let
          \begin{align*}
            \ell_r(x)=\begin{cases}
              1 & \text{if $x \geq r$}, \\
              0 & \text{otherwise},
            \end{cases}
          \end{align*}
          and let $\H=\{\ell_r \mid r\in [0, 1] \}$. Then $\H$ shatters
          $\{0.5\}$, since $\ell_1(0.5)=0$ and $\ell_0(0.5)=1$, but $\H$
          cannot shatter a set of two points, since given $\{p_1, p_2\}$ with
          $p_1 < p_2$, there is no $\ell_r$ such that $\ell_r(p_1) = 1$, but
          $\ell_r(p_2)=0$. Thus, $\VCdim(\H)=1$ despite the fact that $\H$ is
          infinite.
    \item Let $\H$ consist of the constant $0$ and the constant $1$ functions,
          and note that it shatters a singleton set, but cannot shatter a set of
          two points, since it cannot map the two points to different values.
          Hence, $\VCdim(\H)=1=\log_2(|\H|)$.
  \end{enumerate}
\end{ex}

\begin{ex}
\end{ex}

\begin{ex}
  Consider a set of $3$ points, and note that $\H$ shatters it, since given a
  labelling of all ones or all minus ones, we can take the interval that
  includes all three points with the appropriate sign, and any other labelling
  would have only a single one or a single minus one, which we can produce from
  an interval that contains only that point, with the appropriate sign.

  Moreover, $\H$ cannot shatter any set of $4$ points, since given
  $\{p_1,p_2,p_3,p_4\}$, with $p_1<p_2<p_3<p_4$, we cannot produce the labelling
  $(1,-1,1,-1)$, since any positive interval that includes $p_1$ and $p_3$, must
  also include (and thus mislabel) $p_2$, while any negative interval that
  includes $p_2$ and $p_4$ will mislabel $p_3$.

  Hence, $\VCdim(\H)=3$.
\end{ex}

\begin{ex}
\end{ex}

\begin{ex}
\end{ex}

\begin{ex}
\end{ex}